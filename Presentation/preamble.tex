\begin{frame}[noframenumbering,plain]
    \setcounter{framenumber}{1}
    \maketitle
\end{frame}

\begin{frame}
    \frametitle{Положения, выносимые на защиту}
    \begin{itemize}
        \item Впервые осуществлено математическое моделирования тушения лесного пожара при помощи раствора ингибитора,
              распределяемого с помощью взрывного заряда.
        \item На основе численного моделирования показан эффективность совместного действия воды и ингибитора при
              тушении лесного пожара.
        \item Получены оценки влияния массы тушащего состава, доли ингибитора в нем и положения центра инициализации
              противопожарного снаряда на эффективность тушения лесного пожара.
        \item Выявлены физические закономерности, обуславливающие эффективность применения раствора ингибитора при
              тушении лесного пожара.
    \end{itemize}
\end{frame}
\note{
    Проговариваются вслух положения, выносимые на защиту
}

\begin{frame}
    \frametitle{Содержание}
    \tableofcontents
\end{frame}
\note{
    Работа состоит из четырёх глав.

    \medskip
    В первой главе \dots

    Во второй главе \dots

    Третья глава посвящена \dots

    В четвёртой главе \dots
}
